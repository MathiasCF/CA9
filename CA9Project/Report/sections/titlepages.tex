\pdfbookmark[0]{English title page}{label:titlepage_en}
\aautitlepage{%
  \englishprojectinfo{
  	Optimal and Model-Free Reinforcement Learning Control of Water Distribution Networks
   %Machine learning based safe control of water distribution networks %title
  }{%
    9th Semester Project Report %theme
  }{%
    Spring 2022 %project period
  }{%
   932 % project group
  }{%
    %list of group members
    Jeppe Nørgaard Jensen\\
    Mathias Clement Frederiksen
  }{%
    %list of supervisors
    Carsten Skovmose Kallesøe \\
  	Abhijit Mazumdar
  }{%
    1 % number of printed copies
  }{%
     Dec 21, 2022 % date of completion
  }%
}{%department and address
  \textbf{Department of Electronic Systems}\\
  Aalborg University\\
  \href{http://www.aau.dk}{http://www.aau.dk}
}{% the abstract
The focus of this project is development and application of optimal model-free machine learning control methods on a small scale water distribution network test setup with one pumping station, one consumer zone and one elevated water reservoir. Methods examined are; tabular Q-Learning and two versions of function approximation Q-Learning - first with one continuous state and later with two continuous states. A cost function is formulated such that pump power is minimised, while water level in the elevated water reservoir is kept within a certain region such that the chance of dry-out or overflow is minimised and consumer demand is met. Convergence properties of methods are compared in a simulated environment in order to find the most suitable control method candidate. This method is tested on a small scale water distribution network test setup in Aalborg Universities Smart Water Infrastructures Laboratory to verify that the method is applicable as a real world control method.   
}

%\cleardoublepage
%{\selectlanguage{danish}
%\pdfbookmark[0]{Danish title page}{label:titlepage_da}
%\aautitlepage{%
%  \danishprojectinfo{
%    Rapportens titel %title
%  }{%
%    Semestertema %theme
%  }{%
%    Forår 2022 %project period
%  }{%
%    830 % project group
%  }{%
%    %list of group members
%    Laurits Hastrup Andersen\\ 
%	Jeppe Nørgaard Jensen\\
%	Christian Møller Jensen\\
%	Mathias Clement Frederiksen
%  }{%
%    %list of supervisors
%    Carsten Skovmose Kallesøe\\
%	Saruch Satishkumar Rathore \\
%	Roozbeh Izadi-Zamanabadi
%  }{%
%    1 % number of printed copies
%  }{%
%   25. Maj 2022 % date of completion
%  }%
%}{%department and address
%  \textbf{Elektronik og IT}\\
%  Aalborg Universitet\\
%  \href{http://www.aau.dk}{http://www.aau.dk}
%}{% the abstract
%  Her er resuméet
%}}