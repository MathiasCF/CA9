\chapter{Conclusion}\label{chap:Conclusion}
As suggested in \cref{sec:projectDescription} the objective of this project was design and implementation of a optimal and model free  controller for a water distribution network. Three reinforcement learning methods have been applied and compared; tabular Q-learning, continuous height, and discrete time and action function approximation Q-learning, and continuous height and time, and discrete actions function approximation Q-learning. When comparing these three methods it was found that the best control method for real life implementation was continuous height, and discrete time and action function approximation Q-learning due to its ability to generalise, giving better behaviour than the tabular method. The method is also superior to continuous height and time, discrete actions function approximation Q-learning, since this method shows unpredictable and diverging results, and also seems so be extremely hyperparameter sensitive compared to the two other methods. Even though we observe diverging result the method is an interesting candidate for further work since simulations show superior initial performance when comparing moving averages of the cost function. 

Although the examined methods are not compared to existing water distribution network control methods, relative to pumping station energy consumption, we feel confident in saying that the formulated cost function ultimately guarantees minimal energy consumption.

Laboratory testing indicates that continuous height, and discrete time and action function approximation Q-learning is a viable control method that works well, when training is done in a simulated environment. Training in a real life environment was not within the scope of the project, but the implementation lays the foundation for future work and has given us a strong belief that reinforcement learning methods are well suited for real life implementation and can make a difference on both existing and new systems. We note that reinforcement learning control methods developed should include some kind of safety aspect such that safety constraints of systems are not violated. 
